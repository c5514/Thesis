\chapter{Conclusiones y Perspectivas}\label{cap.con}
\markboth{Conclusiones y Perspectivas}{Conclusiones y Perspectivas}
%
Como se ha podido apreciar, el uso de paquetes de onda gaussianos para describir estados de sabor introduce diferencias significativas respecto al tratamiento tradicional con funciones de onda planas, tanto en el marco no relativista como en el relativista.

Una de estas diferencias fundamentales es el comportamiento amortiguado de la probabilidad de transición. En ambos marcos, caso no relativista y relativista, al emplear paquetes gaussianos, esta probabilidad tiende a un valor fijo, $2 \sin^2\theta \cos^2\theta$, para intervalos de tiempo grandes. El intervalo de tiempo necesario para que se llegue de manera aproximada a este valor fijo depende en gran medida de que tan localizada esté la partícula en el espacio de momentos, correspondiendo a una dispersión mayor un menor tiempo de decaimiento. Dado que partículas como los neutrinos y kaones se producen en procesos de colisión, se espera que sus paquetes de onda tengan valores de $\sigma_p$ muy pequeños. Esto implica que, para distancias de propagación suficientemente cortas, la aproximación de onda plana puede seguir siendo una descripción adecuada. Sin embargo, si se desea alcanzar una mayor precisión en la determinación de las masas $m_1$ y $m_2$, las desviaciones introducidas por la naturaleza localizada del estado deben ser consideradas.

Una ventaja conceptual del formalismo de paquetes de onda es que permite un modelado más realista del proceso de detección. Determinar experimentalmente que una partícula creada con sabor $\alpha$ ha oscilado a un sabor $\beta$ implica necesariamente una interacción localizada en una región finita del espacio. En este sentido, no se mide directamente el estado de sabor en un instante, sino que se provoca el colapso de la función de onda a un estado de sabor $\beta$ localizado en la región del detector. Los resultados presentados en este trabajo describen la evolución libre de los estados de sabor, que a diferencia del espín, no interactúan con un campo externo como el campo magnético para su medición. En el caso no relativista, esto puede ser logrado sin problemas, ya que la métrica del espacio de Hilbert de la ecuación de Schrödinger es compatible con la inducida en el subespacio de estados de sabor. En cambio, en el tratamiento relativista basado en la ecuación de Klein-Gordon, esto no es posible pues la densidad de probabilidad no está dada por el módulo al cuadrado de la función de onda. Una solución posible es considerar otra ecuación de movimiento relativista en la que el producto interno sea consistente con el impuesto a los estados de sabor, o la adopción de formulaciones alternativas, como la propuesta por \textcite{Tsai_2016}, en la que se evitan problemas relacionados con las soluciones con energías negativas.

Otra limitación del procedimiento seguido para definir el paquete de onda gaussiano relativista es que se asume como condición inicial la misma función de onda utilizada en el marco no relativista. Esto se debe a que no es válido asumir que la densidad de probabilidad será una distribución gaussiana debido a que la densidad de probabilidad depende también de la primera derivada temporal. Si bien la densidad de probabilidad resultante conserva la noción de localidad alrededor de un punto, no es una distribución gaussiana. Una alternativa viable sería construir el estado inicial imponiendo directamente una distribución gaussiana en el espacio de momentos. Sin embargo, debido a la relación de dispersión relativista \eqref{eq:density_momentum_KG}, el paquete de onda obtenido en el espacio de posiciones diferiría del presentado en este trabajo, ya que la transformada de Fourier incorporaría de manera no trivial el factor $\gamma(p)$.

En resumen, este trabajo ha logrado mostrar las diferencias que introduce el tratamiento con paquetes de onda gaussianos en la oscilación de estados de sabor. Asimismo, se hicieron evidentes las dificultades en la definición de un paquete de onda gaussiano, como la aparición del zitterbewegung y la complicada integral que determina el paquete de onda en el espacio de posiciones, las cuales son resueltas mediante una aproximación.

La perspectiva de trabajo más inmediata que se desprende de este estudio es la extensión del formalismo a la ecuación de Dirac para describir la oscilación de neutrinos, dado que estos son fermiones de espín 1/2. Esta extensión permite evadir el problema de la densidad de probabilidad negativa que surge de las soluciones de la ecuación de Klein-Gordon, pero no evita el zitterbewegung. Por lo que la definición de un paquete de onda gaussiano en este formalismo también tiene dificultades.
