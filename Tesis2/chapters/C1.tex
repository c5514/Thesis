\chapter{Introducción}\label{cap.intro}
% encabezados
\lhead[\thepage]{CAPÍTULO \thechapter. \rightmark}
\rhead[CAPÍTULO \thechapter. \leftmark]{\thepage}
\markboth{INTRODUCCIÓN}{INTRODUCCIÓN}
\pagenumbering{arabic}
\section{Sobre la mecánica relativista}
Desde la propuesta de la relatividad especial en 1905 \cite{einstein1905electrodynamics_english}, se modificó la concepción del espacio-tiempo vigente hasta entonces. Además, con la adopción de la mecánica cuántica como teoría para describir el mundo microscópico y dado que la ecuación de Schrödinger no es compatible con los postulados de la relatividad especial, se inició la búsqueda de ecuaciones relativistas. En primer lugar, a partir de las propuestas de \textcite{Klein1926} y \textcite{Gordon1926}, se obtuvo la ecuación de Klein-Gordon. Sin embargo, al no incluir el espín en su descripción, no era válida para describir el movimiento de los electrones. Posteriormente, \textcite{Dirac1928} propuso otra ecuación relativista, conocida como la ecuación de Dirac, que incorpora el espín en su descripción, siendo válida, en particular, para partículas con espín $1/2$. Más adelante, con el descubrimiento de nuevas partículas elementales, fue necesario encontrar otra ecuación de movimiento que las describiera. Esto llevó a formular la ecuación de Duffin-Kemmer-Petiau, válida para describir la dinámica de mesones con espín 0 y 1 \cite{Petiau1936} \cite{Duffin1938} \cite{Kemmer1939} \cite{Kemmer1943} \cite{SakataTaketani1940}. Esta ecuación generaliza, entre otros, los casos particulares de las ecuaciones de Proca (espín 1) y de Klein-Gordon (espín 0) \cite{Kanatchikov_2000}.
\section{Sobre los paquetes de onda}
En los comienzos de la mecánica cuántica no relativista, una de las primeras propuestas para estudiar la dinámica de una partícula libre fue la de un paquete de onda gaussiano \cite{darwin_propagation}. Esta solución de la ecuación de Schrödinger permitía establecer una relación con la dinámica semiclásica de una partícula. Esfuerzos posteriores llevaron a encontrar otra solución con características similares \cite{airy_functions}, con la diferencia de que esta no se dispersa. Debido al éxito del paquete de onda gaussiano en la descripción del estado de una partícula libre, se intentaron incorporar estas ideas en el contexto de las teorías cuánticas relativistas. Sin embargo, Schrödinger \cite{schrodinger1930forcefree_translation} encontró que en la propagación de un electrón libre existe un efecto conocido como \textit{zitterbewegung}, que se refiere al movimiento tembloroso de la posición media del electrón. Posteriormente, se halló que este fenómeno se debía a la interferencia entre las componentes de energía positiva y negativa que surgen al propagar un paquete de onda \cite{Thaller2005}. Este fenómeno no se limita a las soluciones de la ecuación de Dirac, sino que también aparece en las soluciones de la ecuación de Klein-Gordon \cite{KG_zitter}, y en estas últimas lleva a problemas en la interpretación de las soluciones para el estado de una partícula. Debido a esto se han propuesto modelos para describir la propagación de un paquete de onda gaussiano en donde los efectos del \textit{zitterbewegung} puedan ser despreciados, usando la ecuación de Klein-Gordon \cite{Huang_2024}, pero también ha habido esfuerzos usando otro formalismo que permita evitar estos problemas directamente \cite{Tsai_2016}.
\section{Sobre la oscilación en la física de partículas}
En 1927, Ellis y Wooster realizaron un experimento en el que se obtuvo que la energía promedio de los electrones producidos en el decaimiento beta era menor que la energía total liberada, lo que sugería la existencia de una partícula neutra muy pequeña \cite{EllisWooster1927}. Esta es la partícula llamada neutrino. Aunque en ese momento aún no se tenía una evidencia experimental directa de su existencia, pues debido a que tiene una sección de choque muy pequeña, se había considerado indetectable durante muchos años \cite{Bilenky2018}. En 1946, Pontecorvo propuso un método de detección de neutrinos basado en la reacción:
\begin{equation}
	\ch{$\nu$_e} + \ch{^{37}Cl}\longrightarrow \ch{^{37}Ar} + \ch{e^-},
\end{equation}
como fue mencionado posteriormente en \textcite{Cleveland-1998}, aunque durante esos años no hubo una medición directa de los neutrinos.

En 1947, Rochester y Butler observaron por primera vez trazas en forma de ``V'' en cámaras de niebla, evidenciando la existencia de una nueva partícula neutra e inestable, a la que llamaron mesón V \cite{RochesterButler1947}. Poco después, en 1949, observó el decaimiento:
\begin{equation}
	\ch{$\tau$^+} \longrightarrow \ch{$\pi$^+} + \ch{$\pi$^+} + \ch{$\pi$^-}.
\end{equation}
Esta fue llamada el mesón $\tau$\cite{Brown1949}. Además, se detectó la partícula con decaimiento:
\begin{equation}
	\ch{$\theta$^+} \longrightarrow \ch{$\pi$^+} + \ch{$\pi$^0},
\end{equation}
llamada mesón $\theta$. Estas partículas eran indistinguibles, a menos que se considerará sus modos de decaimiento, pues sus masas y sus tiempos de vida eran idénticos. En 1953, Dalitz obtuvo que si se conservaba la paridad, entonces estas partículas eran diferentes, llevando así al puzzle de $\tau-\theta$ \cite{Aitchison_2006}. Este fue resuelto finalmente en 1956 por Lee y Yang, los cuales propusieron la ruptura de la paridad en la interacción débil \cite{Lee1956}. Debido a que $\tau^+$ y $\theta^+$ son las mismas partículas, la partícula pasó a ser llamada $K^+$.

En 1955, Gell-Mann y Pais analizaron las propiedades del sistema de kaones neutros $K^0$ y $\bar{K}^0$. Basándose en consideraciones de simetría CP, predijeron que los estados físicos que se propagan no son $K^0$ o $\bar{K}^0$ individuales, sino dos combinaciones lineales de los estados $K_1^0$ y $K_2^0$, con tiempos de vida distintos. Esta mezcla de estados implica que un kaón neutro producido como $K^0$ puro evolucionará en el tiempo hacia una superposición de $K^0$ y $\bar{K}^0$ \cite{Gell-Mann-1955}. Este es el primer fenómeno de oscilación de partículas libres.

Pontecorvo, inspirado en el trabajo de Gell-Mann y Pais, sugirió la posibilidad de que existan otras partículas neutras que no sean idénticas a sus antipartículas, es decir, que sea posible una transición entre estos estados \cite{Pontecorvo:1957vz}\cite{Pontecorvo:1957cp}. Estas partículas eran los neutrinos, aunque para ese momento, en el año 1957, se creía que su masa era nula \cite{Abdu-1957}, por lo que las transiciones consideradas eran entre partícula y antipartícula. No fue hasta después del descubrimiento de los neutrinos muónicos \cite{Danby-1962}, por el cual Lederman, Schwartz y Steinberger ganaron el premio Nobel de Física en 1988, que Pontecorvo propuso la posibilidad de una transición entre $\nu_\text{e}$ y $\nu_\mu$ \cite{Pontecorvo:1967fh}.

Posteriormente, Bahcall y Davis realizaron en 1976 un experimento para estudiar la emisión de neutrinos electrónicos producidos en el Sol, conocido como el experimento Homestake. De las muchas reacciones que ocurren en este, se tomó en cuenta principalmente la reacción
\begin{equation}
	\ch{p} + \ch{p} \longrightarrow \ch{^2H} + \ch{e^+} + \ch{$\nu$_e},
\end{equation}
la cual es la mayor contribuyente de los neutrinos electrónicos producidos en el Sol, siendo el $99,75\%$ de todas estas emisiones \cite{Bahcall-1976}.

Para determinar el flujo de neutrinos electrónicos, Davis propuso usar la reacción propuesta por Pontecorvo. Como resultado de este experimento se encontró una diferencia entre el flujo de neutrinos electrónicos detectado y el valor esperado por el modelo estándar solar. Este problema fue atribuido a la eficiencia del experimento, lo cual llevó a múltiples experimentos para verificarlo, como el experimento Kamiokande \cite{Fukuda-1996} y Super-Kamiokande \cite{Hosaka-2006}, en los cuales los neutrinos electrónicos producidos en el Sol eran detectados mediante la colisión elástica
\begin{equation}
	\ch{$\nu$_x} + \ch{e^-} \longrightarrow \ch{$\nu$_x}+ \ch{e^-},
\end{equation}
donde $\ch{x}$ denota que es válida para todo sabor de neutrino.
Sin embargo, se mantuvo este déficit en el flujo de neutrinos solares detectados. Pontecorvo sugirió que este fenómeno era evidencia de la oscilación de neutrinos \cite{Bilenky1977}.

Más adelante, en el experimento realizado en el Observatorio de Neutrinos de Sudbury en el año 2003, conocido como el experimento SNO, se tomaron en cuenta tres reacciones \cite{PhysRevC.75.045502}:
\begin{equation}
	\begin{aligned}[b]
		\ch{$\nu$_e} + \ch{d}   & \longrightarrow \ch{p} + \ch{p} + \ch{e^-},     \\
		\ch{$\nu$_x} + \ch{d}   & \longrightarrow \ch{p} + \ch{n} + \ch{$\nu$_x}, \\
		\ch{$\nu$_x} + \ch{e^-} & \longrightarrow \ch{$\nu$_x} + \ch{e^-},
	\end{aligned}
\end{equation}
donde en la segunda y tercera reacción $\ch{x}$ denota que es válida para cualquier neutrino. En este experimento se obtuvo como resultado que el flujo de neutrinos electrónicos era menor que el predicho por el modelo estándar solar, tal como se obtuvo en experimentos anteriores. Mas este no fue el único resultado, pues de la segunda y tercera reacción se obtuvieron los flujos de los neutrinos muónicos y tauónicos, con los cuales, y junto al flujo de neutrinos electrónicos, se obtuvo el flujo total de neutrinos, el cual era cercano a lo predicho por el modelo estándar solar. Esto implicó que los neutrinos cambiaban su naturaleza al propagarse, por lo que fue una evidencia de la oscilación de neutrinos sugerida por Pontecorvo, e impulsó el desarrollo del estudio de los neutrinos.

Actualmente se sabe que los neutrinos tienen masa, y que sus estados de sabor son una combinación de estados de masa, siendo posible la transición entre estos mediante la matriz PMNS, nombrada en honor a Pontecorvo, Maki, Nakagawa y Sakata por sus contribuciones al desarrollo de la teoría de oscilación de neutrinos, siendo el aporte de estos últimos realizado en el año 1962 \cite{Maki-1962}. Dado que la masa de los neutrinos se determina a partir de la oscilación de neutrinos, este tema sigue siendo de interés en la actualidad \cite{ghosh2023presentstatusfutureprospects}.
\section{Sobre el trabajo}
Para tener una mejor descripción de la oscilación de sabores, se considera utilizar paquetes de onda gaussianos. Este enfoque es más realista que el tratamiento estándar con ondas planas, ya que describe partículas localizadas en el espacio. Dado que los kaones y los neutrinos son partículas relativistas, se debe considerar las ecuaciones relativistas correspondientes. Sin embargo, debido a la complejidad del estudio de las soluciones de paquetes de onda gaussianos en el contexto de las soluciones de la ecuación de Dirac, en este trabajo nos limitaremos a trabajar con funciones de onda escalares, es decir, con aquellas que son soluciones de la ecuación de Klein-Gordon. También se analizará el caso del paquete de onda gaussiano en el marco no relativista, para realizar una comparación entre ambos resultados, y como guía para la construcción del caso relativista. Por esto, en el capítulo 2 se realiza un estudio de las soluciones de las ecuaciones de Schrödinger y Klein-Gordon, y sus densidades de probabilidad asociadas. Además, se lleva a cabo la deducción de la ecuación de Klein-Gordon y se comprueba su invariancia frente a transformaciones de Lorentz. En el capítulo 3 se deriva el paquete de onda gaussiano no relativista y se estudia la propagación de su función de onda y densidad de probabilidad. También se desarrolla el paquete de onda gaussiano en el marco relativista y, usando los resultados de \textcite{Huang_2024}, se realiza una aproximación en la que el zitterbewegung es despreciable. Luego, habiendo definido el paquete de onda gaussiano en ambos formalismos, en el capítulo 4 se introducen los estados de sabor como una combinación lineal de estados de masa, y se estudia la evolución de una partícula que inicialmente está en un estado de sabor, lo que lleva a la oscilación de estados de sabor. Con esto se pueden ver las diferencias cuando se consideran ondas planas y paquetes de onda gaussianos en el estudio de la oscilación de estados de sabor.
