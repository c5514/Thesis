\chapter*{Resumen} % si no queremos que añada la palabra "Capitulo"
\addcontentsline{toc}{chapter}{Resumen}% si queremos que aparezca en el índice
\markboth{RESUMEN}{RESUMEN} % encabezado
\lhead[\thepage]{\rightmark}
\rhead[\leftmark]{\thepage}
\pagenumbering{Roman} %comenzar la numeracion de paginas en numeros romanos
\setcounter{page}{3}

El objetivo de este trabajo es estudiar la oscilación de los estados de sabor utilizando paquetes de onda gaussianos, y mostrar sus diferencias con respecto al método usual basado en funciones de onda planas. El proceso consiste en un estudio paralelo del caso no relativista (ecuación de Schrödinger) y del caso relativista (ecuación de Klein-Gordon), de modo que los resultados del caso relativista se obtengan inspirándose en lo realizado en el caso no relativista, donde no existen problemas asociados a las soluciones de ecuaciones relativistas, como el zitterbewegung y las soluciones de energía negativa. Además, mediante una aproximación validada por el trabajo de \cite{Huang_2024}, se desprecian dichos efectos en el paquete de onda gaussiano relativista. Como resultado de este estudio, se obtiene que la localización de una partícula alrededor de un punto no puede omitirse en el estudio de la oscilación de paquetes de onda, incluso si la densidad de probabilidad asociada a la partícula tiende a la de una función de onda plana conforme se propaga. Esto se debe a que, al considerar paquetes de onda gaussianos, aparece un término de amortiguamiento en la probabilidad de transición entre estados de sabor.
