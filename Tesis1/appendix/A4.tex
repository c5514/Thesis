\chapter{Relación de ortogonalidad de los biespinores $u^{(h)}(p)$ y $v^{(h)}(p)$}\label{aped.D}
\markboth{Relación de ortogonalidad de los biespinores $u^{(h)}(p)$ y $v^{(h)}(p)$}{Relación de ortogonalidad de los biespinores $u^{(h)}(p)$ y $v^{(h)}(p)$}
\lhead[\thepage]{APÉNDICE \thechapter. \rightmark}
\rhead[APÉNDICE \thechapter. \leftmark]{\thepage}
Sabemos que al aplicar la ecuación de Dirac en los biespinores $u^{(h)}(p)$ y $v^{(h)}(p)$ se obtiene que estos satisfacen las ecuaciones \eqref{eq:diracPu(h)} y \eqref{eq:diracPv(h)}, es decir 
\begin{align}
	(\gamma^\mu p_\mu -m)u^{(h)}(p) = 0, \\
	(\gamma^\mu p_\mu +m)v^{(h)}(p) = 0.
\end{align}
Al definir la adjunta de $u^{(h)}(p)$ como
\begin{equation}
  \overline{u^{(h)}}(p)\equiv u^{(h)^\dagger}(p)\gamma^0
\end{equation}
obtendremos de \eqref{eq:diracPu(h)} que
\begin{equation}
	\begin{aligned}[b]
		0 & = u^{(h)^\dagger}(p)(\gamma^{\mu^\dagger}p_\mu -m)\gamma^0 = u^{(h)^\dagger}(p)\gamma^0\gamma^0(\gamma^{\mu^\dagger}p_\mu -m)\gamma^0 \\
		  & = \overline{u^{(h)}}(p)(\gamma^\mu p_\mu -m).\label{eq:diracPu(h)adj}
	\end{aligned}
\end{equation}
De igual forma para
\begin{equation}
  \overline{v^{(h)}}(p) \equiv v^{(h)^\dagger}(p)\gamma^0
\end{equation}
se obtiene de \eqref{eq:diracPv(h)} que
\begin{equation}
	\begin{aligned}[b]
		0 & = v^{(h)^\dagger}(p)(\gamma^{\mu^\dagger}p_\mu +m)\gamma^0 = v^{(h)^\dagger}(p)\gamma^0\gamma^0(\gamma^{\mu^\dagger}p_\mu +m)\gamma^0 \\
		  & = \overline{v^{(h)}}(p)(\gamma^\mu p_\mu +m).\label{eq:diracPv(h)adj}
	\end{aligned}
\end{equation}
Si realizamos el cambio $(p^0,\vec{p}\,)\to (p^0, - \vec{p}\,)$ en las ecuaciones \eqref{eq:diracPu(h)}, \eqref{eq:diracPv(h)}, y denotamos $p_P \equiv(p^0, -\vec{p}\,)$, obtendremos al aplicar $\gamma^0$ por la izquierda que
\begin{equation}
	0 = \gamma^0(\gamma^0p_0 - \gamma^k p_k -m)u^{(h)}(p_P) = (\gamma^\mu p_\mu - m)\gamma^0u^{(h)}(p_P)\label{eq:invDiracu(h)}
\end{equation}
y
\begin{equation}
	0 = (\gamma^0p_0 - \gamma^k p_k + m)v^{(h)}(p_P)  = (\gamma^\mu p_\mu + m)\gamma^0v^{(h)}(p_P)\label{eq:invDiracv(h)}
\end{equation}

Considerando la ecuación de autovalores de la helicidad dadas por \eqref{eq:helicityu(h)} y \eqref{eq:helicityv(h)}, es decir 
\begin{align}
	\frac{\vec{p}\cdot \vec{\Sigma}}{\abs{\vec{p}\,}} u^{(h)}(p) & = h u^{(h)}(p),\\
	\frac{\vec{p} \cdot \vec{\Sigma}}{\abs{\vec{p}\,}}v^{(h)}(p) & = -h v^{(h)}(p),
\end{align}
obtendremos que 
\begin{equation}
		h\overline{u^{(h)}}(p) = u^{(h)^\dagger}(p)\frac{\vec{p}\cdot \vec{\Sigma}^\dagger}{\abs{\vec{p}\,}}\gamma^0 = u^{(h)^\dagger}(p)\gamma^0\frac{\vec{p}\cdot (\gamma^0\vec{\Sigma}^\dagger\gamma^0)}{\abs{\vec{p}\,}}=\overline{u^{(h)}}(p)\frac{\vec{p}\cdot \vec{\Sigma}}{\abs{\vec{p}\,}}
\end{equation}
y
\begin{equation}
		-h\overline{v^{(h)}}(p) = v^{(h)^\dagger}(p)\frac{\vec{p}\cdot \vec{\Sigma}^\dagger}{\abs{\vec{p}\,}}\gamma^0 = v^{(h)^\dagger}(p)\gamma^0\frac{\vec{p}\cdot (\gamma^0\vec{\Sigma}^\dagger\gamma^0)}{\abs{\vec{p}\,}} = \overline{v^{(h)}}(p)\frac{\vec{p}\cdot \vec{\Sigma}}{\abs{\vec{p}\,}}.
\end{equation}
Así como para la ecuación obtenida de la ecuación de Dirac, al aplicar la transformación $(p^0, \vec{p}) \to (p^0,-\vec{p}\,)$ sobre la ecuación de autovalores del operador helicidad obtendremos:
\begin{equation}
	\begin{aligned}[b]
		-\frac{\vec{p}\cdot \vec{\Sigma}}{\abs{\vec{p}}}u^{(-h)}(p_P)                            & = -hu^{(-h)}(p_P)                                    \\
		-\gamma^0 \frac{\vec{p}\cdot(\gamma^0 \vec{\gamma}\gamma^5)}{\abs{\vec{p}}}u^{(-h)}(p_P) & = -h\gamma^0 u^{(-h)}(p_P)                           \\
		\frac{\vec{p}\cdot \vec{\Sigma}}{\abs{\vec{p}}}\gamma^0u^{(-h)}(p_P)                     & = h\gamma^0u^{(-h)}(p_{P})\label{eq:invHelicityu(h)}
	\end{aligned}
\end{equation}
y
\begin{equation}
	\begin{aligned}[b]
		-\frac{\vec{p}\cdot \vec{\Sigma}}{\abs{\vec{p}}}v^{(-h)}(p_P)                            & = -hv^{(-h)}(p_P)                                   \\
		-\gamma^0 \frac{\vec{p}\cdot(\gamma^0 \vec{\gamma}\gamma^5)}{\abs{\vec{p}}}v^{(-h)}(p_P) & = -h\gamma^0 v^{(-h)}(p_P)                          \\
		\frac{\vec{p}\cdot \vec{\Sigma}}{\abs{\vec{p}}}\gamma^0v^{(-h)}(p_P)                     & = h\gamma^0v^{(-h)}(p_P).\label{eq:invHelicityv(h)}
	\end{aligned}
\end{equation}
Notamos que
\begin{equation}
	\begin{aligned}[b]
		\overline{u^{(h)}}(p)\left(\frac{\vec{p}\cdot \vec{S}}{\abs{\vec{p}}}u^{(h')}(p)\right) - \left(\overline{u^{(h)}}(p)\frac{\vec{p}\cdot \vec{S}}{\abs{\vec{p}}}\right)u^{(h')}(p) & = 0 \\
		(h' - h) \overline{u^{(h)}}(p)u^{(h')}(p)                                                                                                                                         & = 0
	\end{aligned}
\end{equation}
implica que $\overline{u^{(h)}}(p)u^{(h')}(p)\propto \delta_{h,h'}$. De igual forma obtenemos que $\overline{v^{(h)}}(p)u^{(h')}(p)\propto \delta_{h,h'}$. Ahora proponemos que
\begin{align}
	\overline{u^{(h)}}(p)u^{(h')}(p) & = 2m\delta_{h,h'},\label{eq:u(h)orth}  \\
	\overline{v^{(h)}}(p)u^{(h')}(p) & = -2m\delta_{h,h'}.\label{eq:v(h)orth}
\end{align}
A partir de estas relaciones hallamos
\begin{equation}
	\begin{aligned}[b]
		\overline{u^{(h)}}(p)\gamma^\mu u^{(h')}(p) & = \frac{1}{2m}\overline{u^{(h)}}(p)\gamma^\mu mu^{(h')}(p) + \frac{1}{2m}\overline{u^{(h)}}(p)m \gamma^\mu u^{(h')}(p)                               \\
		                                            & = \frac{1}{2m}\overline{u^{(h)}}(p)\gamma^\mu\gamma^\nu p_\nu u^{(h')}(p) + \frac{1}{2m}\overline{u^{(h)}}(p)\gamma^\nu p_\nu \gamma^\mu u^{(h')}(p) \\
		                                            & = \frac{1}{m}g^{\mu\nu}p_\nu\overline{u^{(h)}}(p)u^{(h')}(p)                                                                                         \\
		                                            & = \frac{p^{\mu}}{m}\overline{u^{(h)}}(p)u^{(h')}(p)                                                                                                  \\
		                                            & = 2p^\mu \delta_{h,h'}
	\end{aligned}
\end{equation}
\begin{equation}
	\begin{aligned}[b]
		\overline{v^{(h)}}(p)\gamma^\mu v^{(h')}(p) & = \frac{1}{2m}\overline{v^{(h)}}(p)\gamma^\mu mv^{(h')}(p) + \frac{1}{2m}\overline{v^{(h)}}(p)m \gamma^\mu v^{(h')}(p)                               \\
		                                            & = \frac{1}{2m}\overline{v^{(h)}}(p)\gamma^\mu\gamma^\nu p_\nu u^{(h')}(p) + \frac{1}{2m}\overline{v^{(h)}}(p)\gamma^\nu p_\nu \gamma^\mu u^{(h')}(p) \\
		                                            & = \frac{1}{m}g^{\mu\nu}p_\nu\overline{v^{(h)}}(p)v^{(h')}(p)                                                                                         \\
		                                            & = \frac{p^{\mu}}{m}\overline{v^{(h)}}(p)v^{(h')}(p)                                                                                                  \\
		                                            & = 2p^\mu \delta_{h,h'}
	\end{aligned}
\end{equation}
\begin{equation}
	\begin{aligned}[b]
		\overline{u^{(h)}}(p)\gamma^5 u^{(h')}(p) & = \frac{1}{2m}\overline{u^{(h)}}(p)\gamma^5 mu^{(h')}(p) + \frac{1}{2m}\overline{u^{(h)}}(p)\gamma^5 mu^{(h')}(p)                \\
		                                          & = \frac{1}{2m}\overline{u^{(h)}}(p)\gamma^5 mu^{(h')}(p) + \frac{1}{2m}\overline{u^{(h)}}(p)\gamma^5\gamma^\mu p_\mu u^{(h')}(p) \\
		                                          & = \frac{1}{2m}\overline{u^{(h)}}(p)\gamma^5 mu^{(h')}(p) - \frac{1}{2m}\overline{u^{(h)}}(p)\gamma^\mu\gamma^5 p_\mu u^{(h')}(p) \\
		                                          & = \frac{1}{2m}\overline{u^{(h)}}(p)\gamma^5 mu^{(h')}(p) -\frac{1}{2m}\overline{u^{(h)}}(p)\gamma^5 mu^{(h')}(p)                 \\
		                                          & = 0
	\end{aligned}
\end{equation}
\begin{equation}
	\begin{aligned}[b]
		\overline{v^{(h)}}(p)\gamma^5 v^{(h')}(p) & = \frac{1}{2m}\overline{v^{(h)}}(p)\gamma^5 mv^{(h')}(p) + \frac{1}{2m}\overline{v^{(h)}}(p)\gamma^5 mu^{(h')}(p)                \\
		                                          & = \frac{1}{2m}\overline{v^{(h)}}(p)\gamma^5 mv^{(h')}(p) + \frac{1}{2m}\overline{v^{(h)}}(p)\gamma^5\gamma^\mu p_\mu v^{(h')}(p) \\
		                                          & = \frac{1}{2m}\overline{v^{(h)}}(p)\gamma^5 mv^{(h')}(p) - \frac{1}{2m}\overline{v^{(h)}}(p)\gamma^\mu\gamma^5 p_\mu v^{(h')}(p) \\
		                                          & = \frac{1}{2m}\overline{v^{(h)}}(p)\gamma^5 mv^{(h')}(p) -\frac{1}{2m}\overline{v^{(h)}}(p)\gamma^5 mv^{(h')}(p)                 \\
		                                          & = 0
	\end{aligned}
\end{equation}
\begin{equation}
	\begin{aligned}[b]
		u^{(h)^\dagger}(p)v^{(h')}(p_P) & = \overline{u^{(h)}}(p)\gamma^0 v^{(h')}(p_P)                                                                                                                            \\
		                                & = \overline{u^{(h)}}(p)\left(\frac{\gamma^\mu p_\mu}{m}\right)\gamma^0v^{(h')}(p_P) + \overline{u^{(h)}}(p)\left(\frac{-\gamma^\mu p_\mu}{m}\right)\gamma^0v^{(h')}(p_P) \\
		                                & = 0
	\end{aligned}
\end{equation}
\begin{equation}
	\begin{aligned}[b]
		v^{(h)^\dagger}(p)u^{(h')}(p_P) & = \overline{v^{(h)}}(p)\gamma^0 u^{(h')}(p_P)                                                                                                                            \\
		                                & = \overline{v^{(h)}}(p)\left(\frac{\gamma^\mu p_\mu}{m}\right)\gamma^0u^{(h')}(p_P) + \overline{v^{(h)}}(p)\left(\frac{-\gamma^\mu p_\mu}{m}\right)\gamma^0u^{(h')}(p_P) \\
		                                & = 0
	\end{aligned}
\end{equation}
