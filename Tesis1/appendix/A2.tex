\chapter{Propiedades de las matrices gamma}\label{aped.B}
\markboth{Propiedades de las matrices gamma}{Propiedades de las matrices gamma}
\lhead[\thepage]{APÉNDICE \thechapter. \rightmark}
\rhead[APÉNDICE \thechapter. \leftmark]{\thepage}
Las matrices gamma aparecen en la ecuación de Dirac y siguen el álgebra
\begin{equation}
	\left\{\gamma^\mu, \gamma^\nu\right\} = g^{\mu\nu}.
\end{equation}
Además la adjunta de estas satisface la relación
\begin{equation}
	\gamma^0 \gamma^{\mu^\dagger}\gamma^0 = \gamma^\mu
\end{equation}
de la que se obtienen
\begin{equation}
	(\gamma^0)^\dagger = \gamma^0,\quad (\gamma^k)^\dagger = - \gamma^k.
\end{equation}
En este capítulo se desarrollarán algunas de sus propiedades en los espacio-tiempo 3+1 y 2+1 dimensional con el fin de hallar sus posibles representaciones matriciales.

\section{Álgebra de las matrices gamma}
\lhead[\thepage]{\thesection. Álgebra de las matrices gamma}\label{B.1.}
A partir del álgebra de las matrices gamma se puede generar el grupo
\begin{equation}
	\begin{aligned}[c]
		G_{3+1} = & \{\pm I, \pm\gamma^0, \pm\gamma^1, \pm\gamma^2, \pm\gamma^3, \pm\gamma^0\gamma^1, \pm\gamma^0\gamma^2, \pm\gamma^0\gamma^3, \pm\gamma^1\gamma^2, \pm\gamma^1\gamma^3,             \\
		          & \;\pm\gamma^2\gamma^3, \pm\gamma^0\gamma^1\gamma^2, \pm\gamma^0\gamma^1\gamma^3, \pm\gamma^0\gamma^2\gamma^3, \pm\gamma^1\gamma^2\gamma^3, \pm\gamma^0\gamma^1\gamma^2\gamma^3\}.
	\end{aligned}
\end{equation}
Se pueden definir otros operadores los cuales serán miembros de este grupo:
\begin{itemize}
	\item Definimos $\gamma^5$ como
	      \begin{equation}
		      \gamma^5 \equiv \gamma_5 = i\gamma^0\gamma^1\gamma^2\gamma^3.
	      \end{equation}
	      De las propiedades de anticommutación hallamos
	      \begin{equation}
		      \begin{aligned}[c]
			      \left\{\gamma^5, \gamma^0\right\} = i \left\{\gamma^0\gamma^1\gamma^2\gamma^3, \gamma^0\right\} = i\gamma^0\gamma^1\gamma^2\gamma^3\gamma^0 + i \gamma^0\gamma^0\gamma^1\gamma^2\gamma^3 = 0 \\
			      \left\{\gamma^5, \gamma^1\right\} = i \left\{\gamma^0\gamma^1\gamma^2\gamma^3, \gamma^1\right\} = i\gamma^0\gamma^1\gamma^2\gamma^3\gamma^1 + i \gamma^1\gamma^0\gamma^1\gamma^2\gamma^3 = 0 \\
			      \left\{\gamma^5, \gamma^2\right\} = i \left\{\gamma^0\gamma^1\gamma^2\gamma^3, \gamma^2\right\} = i\gamma^0\gamma^1\gamma^2\gamma^3\gamma^2 + i \gamma^2\gamma^0\gamma^1\gamma^2\gamma^3 = 0 \\
			      \left\{\gamma^5, \gamma^3\right\} = i \left\{\gamma^0\gamma^1\gamma^2\gamma^3, \gamma^0\right\} = i\gamma^0\gamma^1\gamma^2\gamma^3\gamma^3 + i \gamma^3\gamma^0\gamma^1\gamma^2\gamma^3 = 0 \\
		      \end{aligned}
	      \end{equation}
	      \begin{equation}
		      \begin{aligned}[b]
			      (\gamma^5)^2 & = - \gamma^0\gamma^1\gamma^2\gamma^3\gamma^0\gamma^1\gamma^2\gamma^3 \\
			                   & = -(-1)^4(-1)^2(-1)(\gamma^0)^2(\gamma^1)(\gamma^2)^2(\gamma^3)^2    \\
			                   & = I
		      \end{aligned}
	      \end{equation}
	      \begin{equation}
		      \begin{aligned}[b]
			      (\gamma^5)^\dagger & = -i \gamma^{3^\dagger}\gamma^{2^\dagger}\gamma^{1^\dagger}\gamma^{0^\dagger} \\
			                         & = -i (-\gamma^3)(-\gamma^2)(-\gamma^1)\gamma^0                                \\
			                         & = i(-1)^3(-1)^2(-1)\gamma^0\gamma^1\gamma^2\gamma^3                           \\
			                         & = \gamma^5
		      \end{aligned}
	      \end{equation}
	\item Definiendo $\sigma^{\mu\nu}$ como
	      \begin{equation}
		      \sigma^{\mu\nu} \equiv \frac{i}{2}\left[\gamma^\mu,\gamma^\nu\right]
	      \end{equation}
	      obtenemos las relaciones
	      \begin{equation}
		      \begin{aligned}[b]
			      \left[\gamma^5,\sigma^{\mu\nu}\right] & = \frac{i}{2}\left[\gamma^5, \gamma^\mu\gamma^\nu\right] - \frac{i}{2}\left[\gamma^5, \gamma^\nu\gamma^\mu\right]                                    \\
			                                            & = \frac{i}{2}\left(\gamma^5\gamma^\mu\gamma^\nu - \gamma^\mu\gamma^\nu\gamma^5 - \gamma^5 \gamma^\nu\gamma^\mu + \gamma^\nu\gamma^\mu\gamma^5\right) \\
			                                            & = \frac{i}{2}\left(-\gamma^\mu\gamma^5\gamma^\nu + \gamma^\mu\gamma^5\gamma^\nu + \gamma^\nu\gamma^5\gamma^\mu - \gamma^\nu\gamma^5\gamma^\mu\right) \\
			                                            & = 0,
		      \end{aligned}
	      \end{equation}
	      \begin{equation}
		      \begin{aligned}[b]
			      \gamma^0(\sigma^{\mu\nu})^\dagger\gamma^0 & = \gamma^0\left(\frac{i}{2}\gamma^\mu\gamma^\nu - \frac{i}{2}\gamma^\nu\gamma^\mu\right)^\dagger\gamma^0                                                              \\
			                                                & = \gamma^0\left(-\frac{i}{2}\gamma^{\nu^\dagger}\gamma^{\mu^\dagger} + \frac{i}{2}\gamma^{\mu^\dagger}\gamma^{\nu^\dagger}\right)\gamma^0                             \\
			                                                & = \frac{i}{2}\gamma^0\bigg(-(\gamma^0\gamma^{\nu}\gamma^0)(\gamma^0\gamma^{\mu}\gamma^0) + (\gamma^0\gamma^{\mu}\gamma^0)(\gamma^0\gamma^{\nu}\gamma^0)\bigg)\gamma^0 \\
			                                                & = \frac{i}{2}(\gamma^\mu\gamma^\nu - \gamma^\nu\gamma^\mu)                                                                                                            \\
			                                                & = \sigma^{\mu\nu}.
		      \end{aligned}
	      \end{equation}
	\item Definiendo $\Sigma^k$ como
	      \begin{equation}
		      \Sigma^k \equiv \frac{1}{2}\sum_{j,l}\epsilon^{kjl}\sigma^{jl} = \gamma^0\gamma^k \gamma^5
	      \end{equation}
	      obtenemos que
	      \begin{equation}
		      \begin{aligned}[b]
			      \sum_l\epsilon^{jkl}\Sigma^l & = \frac{1}{2}\sum_{l,m,n}\epsilon^{jkl}\epsilon^{lmn}\sigma^{mn}                                                                         \\
			                                   & = \frac{1}{2}\left(\epsilon^{pjk}\epsilon^{pjk}\sigma^{jk} + \epsilon^{pjk}\epsilon^{pkj}\sigma^{kj}\right)\quad \text{donde }p \neq j,k \\
			                                   & = \sigma^{jk},
		      \end{aligned}
	      \end{equation}
	      \begin{equation}
		      \begin{aligned}[b]
			      \left[\Sigma^k, \Sigma^j\right] & = \left[\gamma^0\gamma^k\gamma^5, \gamma^0\gamma^j\gamma^5\right]                                     \\
			                                      & = \gamma^0\gamma^k\gamma^5\gamma^0\gamma^j\gamma^5 - \gamma^0\gamma^j\gamma^5\gamma^0\gamma^k\gamma^5 \\
			                                      & = -(\gamma^0)^2\gamma^k\gamma^j(\gamma^5)^2 + (\gamma^0)^2\gamma^j\gamma^k(\gamma^5)^2                \\
			                                      & = 2i \sigma^{kj}                                                                                      \\
			                                      & = 2i\sum_l\epsilon^{kjl}\Sigma^l,
		      \end{aligned}
	      \end{equation}
	      \begin{equation}
		      \begin{aligned}[b]
			      \left\{\Sigma^k, \Sigma^j\right\} & = \left\{\gamma^0\gamma^k\gamma^5,\gamma^0\gamma^j\gamma^5\right\}                                    \\
			                                        & = \gamma^0\gamma^k\gamma^5\gamma^0\gamma^j\gamma^5 + \gamma^0\gamma^j\gamma^5\gamma^0\gamma^k\gamma^5 \\
			                                        & = -(\gamma^0)^2\gamma^k\gamma^j(\gamma^5)^2 - (\gamma^0)^2\gamma^j\gamma^k(\gamma^5)^2                \\
			                                        & = 2\delta_{kj},\label{eq:AntiSigma}
		      \end{aligned}
	      \end{equation}
	      \begin{equation}
		      \begin{aligned}[b]
			      (\Sigma^k)^\dagger & = (\gamma^5)^\dagger(\gamma^k)^\dagger(\gamma^0)^\dagger \\
			                         & = -\gamma^5\gamma^k\gamma^0                              \\
			                         & = \gamma^0\gamma^k\gamma^5                               \\
			                         & = \Sigma^k.
		      \end{aligned}
	      \end{equation}
\end{itemize}
\section{Representaciones de las matrices gamma}
Se sabe que para un grupo finito no abeliano $G$ se cumple que
\begin{equation}
	\abs{G} = \sum_{\lambda=1}^N(n_\lambda)^2\label{ap2:dimenRep}
\end{equation}
donde $\abs{G}$ es el número de elementos de $G$, $N$ es el número de representaciones irreducibles inequivalentes de $G$ y $n_\lambda$ es la dimensión de la representación.

Además el número de representaciones irreducibles inequivalentes es el mismo que el número de clases conjugadas. Por ello hallamos las clases conjugadas del grupo $G_{3+1}$:
\begin{equation}
	\begin{aligned}[c]
		\left\{I\right\},\left\{-I\right\},\left\{\gamma^0,-\gamma^0\right\},\left\{\gamma^1,-\gamma^1\right\},\left\{\gamma^2,-\gamma^2\right\},\left\{\gamma^3,-\gamma^3\right\},\left\{\gamma^0\gamma^1,-\gamma^0\gamma^1\right\},                                            \\
		\left\{\gamma^0\gamma^2,-\gamma^0\gamma^2\right\},\left\{\gamma^1\gamma^2,-\gamma^1\gamma^2\right\},\left\{\gamma^1\gamma^3,-\gamma^1\gamma^3\right\},\left\{\gamma^2\gamma^3,-\gamma^2\gamma^3\right\},                                                                 \\
		\left\{\gamma^0\gamma^1\gamma^2,-\gamma^0\gamma^1\gamma^2\right\},\left\{\gamma^0\gamma^1\gamma^3,-\gamma^0\gamma^1\gamma^3\right\},\left\{\gamma^0\gamma^2\gamma^3,-\gamma^0\gamma^2\gamma^3\right\},\left\{\gamma^1\gamma^2\gamma^3,-\gamma^1\gamma^2\gamma^3\right\}, \\
		\left\{\gamma^0\gamma^1\gamma^2\gamma^3,-\gamma^0\gamma^1\gamma^2\gamma^3\right\},
	\end{aligned}
\end{equation}
entonces
\begin{equation}
	32 = \sum_{\lambda=1}^{17}(n_\lambda)^2 = 4^2 + \underbrace{1 + 1\dots + 1}_{\text{$16$ veces}}.
\end{equation}
Esto quiere decir que existe solo una representación irreducibles que no es trivial, la cual tiene dimensión 4.

Tomando esto en cuenta ahora se muestran algunas de las representaciones, las cuales son equivalentes, de las matrices gamma.
\subsection{Representación de Dirac}\label{ssec:DiracRep}
La representación de las matrices $\gamma$ es
\begin{equation}
	\gamma^0 = \begin{pmatrix}
		I & 0  \\
		0 & -I
	\end{pmatrix},\quad \gamma^k = \begin{pmatrix}
		0         & \sigma^k \\
		-\sigma^k & 0
	\end{pmatrix}.
\end{equation}
Entonces el operador $\gamma^5$ es representado como
\begin{equation}
	\gamma^5 = \begin{pmatrix}
		0 & I \\
		I & 0
	\end{pmatrix},
\end{equation}
el operador $\sigma^{\mu\nu}$ como
\begin{equation}
	\sigma^{0k} = i \begin{pmatrix}
		\sigma_k & 0          \\
		0        & - \sigma_k
	\end{pmatrix},\quad \sigma^{kj}= \sum_l \epsilon^{kjl}\begin{pmatrix}
		\sigma_l & 0        \\
		0        & \sigma_l
	\end{pmatrix}
\end{equation}
y el operador $\vec{\Sigma}$ como
\begin{equation}
	\vec{\Sigma} = \begin{pmatrix}
		\vec{\sigma} & 0            \\
		0            & \vec{\sigma}
	\end{pmatrix}.
\end{equation}
\subsection{Representación quiral}\label{ssec:QuiralRep}
\begin{equation}
	\gamma^0 = \begin{pmatrix}
		0  & -I \\
		-I & 0
	\end{pmatrix},\quad \gamma^k = \begin{pmatrix}
		0         & \sigma^k \\
		-\sigma^k & 0
	\end{pmatrix}.
\end{equation}
Entonces el operador $\gamma^5$ es representado como
\begin{equation}
	\gamma^5 = \begin{pmatrix}
		I & 0  \\
		0 & -I
	\end{pmatrix},
\end{equation}
el operador $\sigma^{\mu\nu}$ como
\begin{equation}
	\sigma^{0k} = i \begin{pmatrix}
		\sigma_k & 0        \\
		0        & \sigma_k
	\end{pmatrix},\quad \sigma^{kj} = \sum_l \epsilon^{kjl}\begin{pmatrix}
		\sigma^l & 0        \\
		0        & \sigma^l
	\end{pmatrix}
\end{equation}
y el operador $\vec{\Sigma}$ como
\begin{equation}
	\vec{\Sigma} = \begin{pmatrix}
		\vec{\sigma} & 0            \\
		0            & \vec{\sigma}
	\end{pmatrix}.
\end{equation}
