\chapter{Introducción}\label{cap.intro}
% encabezados
\lhead[\thepage]{CAPÍTULO \thechapter. \rightmark}
\rhead[CAPÍTULO \thechapter. \leftmark]{\thepage}
\markboth{INTRODUCCIÓN}{INTRODUCCIÓN}
\pagenumbering{arabic}

La relatividad especial fue propuesta en 1905 por \textcite{einstein1905electrodynamics_english}, esta modificó los conceptos que se tenían de espacio y tiempo, así como las transformaciones entre dos sistemas inerciales. En vista de este problema, se propuso una ecuación relativista, siendo la primera la ecuación Klein-Gordon \cite{klein1926quantentheorie}\cite{gordon1926quantentheorie}, la cual es usada como base para hallar otras ecuaciones relativistas usando el método de \textcite{umezawa}. En particular, con este método se deduce la ecuación Dirac, la cual fue deducida por Dirac en 1928 y describe la dinámica de partículas con espín 1/2 \cite{dirac1928quantum}. Estas ecuaciones relativistas al formar parte al estar restringidas con la condición de mantener su forma en distintos marcos inerciales, están ligadas a las representación del grupo de Poincaré. Por esto, en el estudio de las ecuaciones relativistas surgen problemas que no se consideraban en la mecánica cuántica no relativistas, esto principalmente debido a que el grupo de Lorentz, el cual es un subgrupo del grupo de Poincaré, tiene cuatro componentes conexas \cite{cobos_zara}. Debido a esto el estudio de la helicidad y la quiralidad es imporante, pues ambas están relacionadas con la transformación de paridad.


La helicidad tiene una de sus aplicaciones en el estudio de la interacción débil. Esto se debe a que la transformación de inversión espacial intercambia el signo del autovalor de la helicidad de una partícula, por lo que puede ser usada para determinar la ``orientación'' dominante \cite{Gartner2015}. De este modo se relaciona a la ruptura de paridad, cuya predicción teórica fue hecha por \textcite{LeeYang1956} en 1956 y confirmada experimentalmente por \textcite{Wu1957} en 1957. Otras de sus aplicaciones es en el estudio del grafeno, donde los electrones alrededor de la intercepción entre las bandas de valencia y conducción pueden ser descritos por la ecuación de Dirac para una partícula sin masa restringida al plano. En este caso, el Hamiltoniano de Dirac es proporcional a la helicidad, pudiendo ser considerados ambos como equivalentes para la descripción del estado de los electrones \cite{B_hmer_2012}.

La quiralidad tiene un rol importante en el estudio del grafeno, donde su descripción a partir del modelo ``tight-binding'' hace que tenga simetría quiral. Similar a lo que ocurre en electrodinámica cuántica, donde la masa rompe la simetría quiral, en el grafeno la interacción coulombiana genera un término de masa efectiva que produce la ruptura de dicha simetría \cite{Semenoff_2012}.

Este trabajo tiene como objetivos encontrar un conjunto completo de observables compatibles, a partir del cual se pueda caracterizar el estado de una partícula libre. Como consecuencia de este objetivo, es necesario el estudio de la helicidad y sus autovalores, de modo que se pueda expresarse la solución general de la ecuación de Dirac en función de ellos. Otro de los objetivos es encontrar la relación entre la quiralidad y la transformación de inversión espacial, y determinar bajo que condición sus autovalores permiten caracterizar el estado de una partícula. Se espera que este desarrollo sirva como base para el desarrollo de temas más complejos en los cuales la helicidad y quiralidad tienen un rol importante, como los mencionados anteriormente.

Este trabajo está organizado de la siguiente manera: En el capítulo 2 se realiza un repaso de los elementos de la mecánica cuántica no relativista con el fin de introducir el concepto del conjunto completo de observables compatibles. Luego se realiza la deducción de la métrica de Minkowski a partir del segundo postulado de la relatividad especial, para así estudiar sus propiedades. De este modo se sigue con la obtención de dos ecuaciones relativistas, siendo estas la ecuación de Klein-Gordon y la ecuación de Dirac. En el capítulo 3 se estudiaron los generadores de los grupos de Lorentz y Poincaré, y se obtuvieron el álgebra de sus generadores. A partir estos generadores se obtienen los operadores momento angular, boost y espín. Además se comprueba que los operadores $W^2$ y $P^2$ son invariantes relativistas, que son utilizados para definir un conjunto completo de observables compatibles con el fin de caracterizar el estado de una partícula. En el capítulo 4 se introduce el operador helicidad y se calculan las relaciones que siguen los biespinores $u^{(h)}(p)$ y $v^{(h)}(p)$. Además, se define el operador quiralidad y se estudia su relación con la transformación de inversión espacial y la helicidad.
