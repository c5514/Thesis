\chapter{Conclusiones y Perspectivas}\label{cap.con}
\markboth{Conclusiones y Perspectivas}{Conclusiones y Perspectivas}
%
% Como hemos podido apreciar \\
%
% Este estudio nos permite tener las siguientes perspectivas:\\

Se obtuvo que los operadores $W^2$, $P^2$, $P^\mu$ y $\hat{h}$ permiten caracterizar a una partícula libre. Además, como $W^2$ y $P^2$ son proporcionales a la identidad no son observables válidos pero son propiedades intrínsecas de la partícula. De esto se sigue que los operadores $P^\mu$ y $\hat{h}$ forma un conjunto completo de observables compatibles, a partir de cuyas autofunciones y relaciones de ortogonalidad se puede determinar cualquier solución de la ecuación de Dirac. Además, la helicidad al depender explícitamente de $\vec{P}$,
\begin{equation}
	\hat{h} \psi(x)  = \frac{\vec{\Sigma}\cdot \vec{P}}{\abs{\vec{P}}}\psi(x),
\end{equation}
un cambio de referencia adecuado puede invertir el signo de la helicidad por lo que no es invariante frente a las transformaciones de Lorentz, así como lo son los invariantes relativistas $W^2$ y $P^2$.

La quiralidad al estar relacionada con la transformación de paridad, permite definir una idea de ``orientación'' en las soluciones de la ecuación de Dirac, la cual no está bien definida, pues sus estas componentes evolucionan por medio de dos ecuaciones diferenciales parciales acopladas. Esto implica que, en general, la quiralidad no es una propiedad de las partículas libres.

Sin embargo, en el caso de una partícula sin masa, las componentes evolucionan independientemente, de modo pudiendo relacionar a la quiralidad con una propiedad de la partícula. Esto se verifica al obtener que en este caso la helicidad y la quiralidad son iguales. Además, al considerarse partículas no masivas, no existe un boost que permita invertir el signo de la helicidad, siendo así la helicidad válidad para cualquier sistema de referencia.

Una continuación de este trabajo es usar el método de Umezawa \cite{umezawa} para obtener otras ecuaciones relativistas, como la ecuación de Duffin-Kemmer-Petiau, y determinar mediante la covariancia de estas ecuaciones la representación del espín y la helicidad. Otro posible desarrollo es estudiar la helicidad y la quiralidad en presencia de interacciones con un campo \cite{Singh_2004}.

Por último, otro tema de interés es evaluar el caso de la ecuación de Dirac en un espacio-tiempo 2+1 dimensional, la cual surge al describir la dinámica alrededor de los conos de Dirac en el grafeno \textcite{Castro_Neto_2009}. En esta caso, es de interés estudiar la quiralidad y la helicidad, y su relación, así como se hizo en un espacio-tiempo 2+1 dimensional.
