\chapter*{Resumen} % si no queremos que añada la palabra "Capitulo"
\addcontentsline{toc}{chapter}{Resumen}% si queremos que aparezca en el índice
\markboth{RESUMEN}{RESUMEN} % encabezado
\lhead[\thepage]{\rightmark}
\rhead[\leftmark]{\thepage}
\pagenumbering{Roman} %comenzar la numeracion de paginas en numeros romanos
\setcounter{page}{3}

En este trabajo se demostrará que la helicidad es una característica de una partícula, la cual depende del sistema de referencia, a menos que la partícula se mueva a la velocidad de la luz. Asimismo, se mostrará que la quiralidad define un sentido de orientación de una partícula cuando esta no tiene masa, y que en este caso es equivalente a la helicidad.

Con este fin se estudian los generadores del grupo de Lorentz y Poincaré, denotados por $J^{\mu\nu}$ y $P^\sigma$, siendo $J^{jk}$ los generadores de rotaciones, $J^{0k}$ los de los boosts y $P^\sigma$ de las traslaciones temporales y espaciales, respectivamente. A partir de estos se definen nuevos generadores, $J^k$, $S^k$ y $L^k$, los cuales satisfacen el mismo álgebra de las componentes del momentum angular de la mecánica cuántica no relativista, identificando así a los operadores de momento angular orbital, espín y momento angular total, respectivamente. Además, se construyen los operadores $P^2$ y $W^2$, los cuales conmutan con todos los generadores mencionados y se relacionan con la masa y el espín de la partícula.

A partir de $W^0$ se define el operador helicidad, y se calculan sus autovalores en la representación de Dirac. Posteriormente, mediante la aplicación de la transformación de inversión espacial a la ecuación de Dirac, se demuestra que las autofunciones del operador quiralidad $\gamma^5$ se intercambian bajo esta transformación. Este resultado permite descomponer cualquier solución de la ecuación de Dirac en dos partes con ``orientación'' diferente que evolucionan como dos ecuaciones diferenciales acopladas, a menos que se considere una partícula sin masa.
